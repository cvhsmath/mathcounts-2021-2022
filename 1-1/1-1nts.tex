\documentclass{article}
\usepackage{enumitem}
\usepackage{siunitx}
\usepackage{amsmath}
\title{Unit 1: Algebra \\ Systems of Equations}
\author{}
\date{}
\begin{document}
    \maketitle
    \section*{Important Concepts}
    \subsection*{Variables, Expressions, and Equations}
    A \textbf{variable} is a letter used to represent a potentially unknown
    value. $x$ is often used by convention to denote a value in an equation
    which we're trying to find. An \textbf{expression} is a combination of
    \emph{symbols}, which can include numbers such as $42$, operators such as
    $+$, and variables such as $x$. An \textbf{equation} is a statement which
    states that two expressions are equal. For example, $2x - 7 = x - 8$ is an
    equation which states that the values of the expression $2x - 7$ and the
    expression $x - 8$ are equal.
    \subsection*{Notation}
    In algebra and more advanced mathematics, the $\times$ sign is rarely used
    to denote multiplication. Instead, the $\cdot$ sign is used. For example, we
    write $3 \cdot 4$ instead of $3 \times 4$. The multiplication operator may
    also be completely omitted if it doesn't cause ambiguity. For example, $ab$
    means $a$ times $b$, and $3x$ means $3$ times $x$. When we multiply a number
    by a variable, we typically write the number first, so we would write $4x$
    instead of $x4$. We also rarely use the $\div$ sign. Instead, we use the $/$
    sign or the fraction bar.
    \subsection*{Substituting Values}
    The value of $2x - 7$ depends on the value of $x$. If we know the value of
    $x$, we can \textbf{substitute} it into the expression to find the value of
    the expression. For example, if $x = 4$, then we can replace the $x$ in $2x
    - 7$ with $4$, so $2x - 7 = 2 \cdot 4 - 7 = 8 - 7 = 1$.
    \subsection*{Solving Equations}
    When we have an equation involving variables, it's often useful to determine
    what values can be assigned to the variables such that the equation is true.
    We say that a value for an variable \textbf{satisfies} an equation if
    substituting that value into the equation results in a true statement, and
    we call such a value a \textbf{solution}.

    If we know that two expressions are equal, we can apply the same operation
    to both sides and they will still be equal. This is an important fact that
    we can use to solve equations. For example, if we know that $x + 2 = 5$,
    then we know that subtracting $2$ from both sides will result in two equal
    values. Therefore $x + 2 - 2 = 5 - 2$, so $x = 3$. By applying the same
    operation to both sides, we can \emph{isolate} the variable so that we can
    determine its value. For more complicated equations, we might need to apply
    multiple operations as shown in the video. If a variable appears multiple
    times, we will need to ``combine'' them by writing addition as
    multiplication or factoring. For example, $2x + 3x = (2 + 3)x = 5x$.
    \subsection*{Systems of Equations}
    An equation with two variables generally has infinitely many solutions. For
    example, some of the solutions of $y = 3x + 5$ are $(x, y) = (0, 5)$, $(x,
    y) = (1, 8)$, and $(x, y) = (2, 11)$. If we add in a second equation
    involving the two variables, we get a \textbf{system of equations} which
    usually have one solution that satisfies both equations. For example, the
    single solution to $y = 3x + 5$ and $2y = x + 9$ is $x = -\frac{1}{5}, y =
    \frac{22}{5}$. To find the solutions to systems of equations, we can use
    \textbf{substitution} and \textbf{elimination}. These methods transform a
    system of two equations into a single equation involving one variable.
    \subsubsection*{Substitution}
    To solve a system of equations by substitution, we isolate one of the
    variables in one equation and substitute the expression into the other
    equations. For example, to solve $y = 3x + 5$ and $2y = x + 9$, we notice
    that in the first equation, $y$ is already isolated. Therefore, we can
    substitute the expression which $y$ is equal to into the second equation to
    get $2(3x + 5) = x + 9$. This can then be solved using the methods for
    single variable equations to find the value of $x$. Once we know $x$, we can
    substitute its value into the first equation to find $y$.
    \subsubsection*{Elimination}
    We know that adding the same value to both sides of an equation results in
    an equivalent equation. If we know that $x + y = 8$ and $x - y = 4$, we can
    add $x - y$ to the left side of the first equation and $4$ to the right side
    since they are equal. Therefore, $x + y + x - y = 8 + 4$, so $2x = 12$ and
    we've eliminated one of the variables. We can then solve for $x$ and use the
    value of $x$ to find $y$. Sometimes, we may have to multiply the equations
    by some constants before adding or subtracting them so that one of the
    variables is eliminated. For example, if we have $x + y = 42$ and $2x + 3y =
    2$, we can first multiply the first equation by $-2$ to get $-2x - 2y = -84$
    and then add the two equations to eliminate $2x$.

    \section*{Example Problem}
    You can model many word problems with a system of equations. 
    In this case, we already have a system set up for you.
    \begin{align}
        x + y &= 6 \label{eq1} \\
        x - y &= 5 \label{eq2}
    \end{align}
    We have a choice of substitution and elimination. We'll first try 
    substititon, and then we'll cover elimination.
    
    \subsection*{Substitution}
    \begin{enumerate}
    \item Add $y$ to the left and right side of the equal sign of equation \eqref{eq2}
    to isolate the $x$ variable to one side.
    \begin{align}
        x - y &= 5 \nonumber \\
        x &= 5 + y \label{eq3}
    \end{align} 
    \item Then, substitute this new equation \eqref{eq3} into equation 
    \eqref{eq1} and simplify.
    \begin{align*}
        x + y &= 6 \\
        (5 + y) + y &= 6 \\
        2y + 5 &= 6 \\
        2y &= 1 \\
        y &= \frac{1}{2}
    \end{align*}
    \item After solving for $y$, we can substititute this equation back in to 
    equation \eqref{eq1} to solve for $x$.
    \begin{align*}
        x + y &= 6 \\
        x + \frac{1}{2} &= 6 \\
        x &= \frac{11}{2}
    \end{align*} 
    \item We can check our answers by plugging in these $x$ and $y$ values 
    into equation \eqref{eq2}.
    \begin{align*}
        x + y &= 6 \\
        \frac{11}{2} + \frac{1}{2} &= 6
    \end{align*}
    \end{enumerate}

    \subsection*{Elimination}
    \begin{enumerate}
        \item Add equations \eqref{eq1} and \eqref{eq2} together and simplify. 
        Notice that we are left with a single-variable equation.
        \begin{align*}
            x + y &= 6 \nonumber \\
            x - y &= 5 \nonumber \\
            (x + y) + (x - y) &= 6 + 5 \\
            2x &= 11 \\
            x &= \frac{11}{2}
        \end{align*}
        \item Substitute the $x$ value into equation \eqref{eq1} to solve for $y$.
        \begin{align*}
            x + y &= 6 \\
            \frac{11}{2} + y &= 6 \\
            y &= \frac{1}{2}
        \end{align*}
        \item Check your work by plugging in the $x$ and $y$ values into another equation.
    \end{enumerate}
\end{document}