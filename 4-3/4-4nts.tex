\documentclass{article}

\usepackage[margin=0.5in]{geometry}
\usepackage{multicol}
\usepackage{siunitx}
\usepackage{hyperref}

\title{4.3 Solid Geometry Notes}
\author{}
\date{}

\begin{document}
\maketitle
\begin{multicols}{2}
	\section*{Terminology}
	A \textbf{polyhedron} is basically a three-dimensional shape formed by flat polygon \textbf{faces}.
	The straight lines that form the borders of the polygon faces are called \textbf{edges} and the points that are at the intersection of edges are called \textbf{vertices}.
	A \textbf{convex} polyhedron is a polyhedron that only bends inwards, similar to how a convex polygon only contains interior angles less than $\ang{180}$.

	\section*{Euler's Formula}
	In a convex polyhedron, if $V$ is the number of vertices, $E$ is the number of edges, and $F$ is the number of faces, then $V - E + F = 2$.
	The proof is on Wikipedia (\url{https://en.wikipedia.org/wiki/Euler_characteristic#Proof_of_Euler's_formula}) if you're interested.
	
	\section*{Volume and Surface Area Formulas}
	\subsection*{Prisms}
	A prism is a polyhedron comprising an $n$-sided polygonal base, a second base which is a translated copy (rigidly moved without rotation) of the first, and $n$ other faces (necessarily all parallelograms) joining corresponding sides of the two bases.\footnote{Shamelessly copied from \href{https://en.wikipedia.org/wiki/Prism_(geometry)}{Wikipedia}, CC BY-SA 3.0.}
	The volume of a prism is the product of the area of its base and its height.
	The surface area of a prism can be found by finding the area of each face and adding everything up.

	\subsection*{Cylinders}
	The volume of a cylinder can be found by multiplying its base area with its height, just like prisms.
	To find the surface area of a cylinder, imagine unwrapping its side surface into a rectangle.
	Its height is the height of the cylinder and its width is the circumference of its base.
	Take the product of the base circumference and the height, and add it to the area of the two circular faces.

	\subsection*{Pyramids}
	The volume of a pyramid is a third of the volume of a prism with the same base and height.

	\subsection*{Cones}
	Similar to pyramids, the volume of a cone is a third of the volume of the cylinder with the same base and height.

	\subsection*{Spheres}
	The volume of a sphere with radius $r$ is $\frac{4}{3}\pi r^3$.
	The surface area of a sphere is $4\pi r^2$, which is four times the area of a circle with the same radius.
	Here's a fascinating proof of this formula from 3Blue1Brown: \url{https://youtu.be/GNcFjFmqEc8}.
\end{multicols}
\end{document}
