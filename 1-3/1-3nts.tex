\documentclass{article}

\usepackage{multicol}
\usepackage{siunitx}
\usepackage{enumitem}
\usepackage{amsmath}

\title{1.3 Arithmetic Notes}
\author{}
\date{}

\begin{document}
\maketitle
\subsection*{Fractions}
    \[\frac{a}{c} + \frac{b}{c} = \frac{a + b}{c}\]
    \[\frac{a}{c} + \frac{b}{d} = \frac{a}{c} \left(\frac{d}{d}\right) + \frac{b}{d}\biggl(\frac{c}{c}\biggl) = \frac{ad + bc}{cd}\]
    \[\frac{1}{a} + \frac{1}{b} = \frac{1}{a} \left(\frac{b}{b}\right) + \frac{1}{b}\biggl(\frac{a}{a}\biggl) = \frac{b + a}{ab}\] 

\subsection*{Exponent and Radical Rules}
    \[x^a \cdot x^b = x^{(a + b)}\]
    \[(xy)^a = x^a y^a\]
    \[(x^a)^b = x^{ab}\]
    \[x^{\frac{a}{b}} = \sqrt[b]{x^a} = \left(\sqrt[b]{x}\right)^a\]
    \[x^{-a} = \frac{1}{x^a}\]
    \[\sqrt{ab} = \sqrt{a}\sqrt{b}\]
    \[(x + y)^a \neq x^a + y^a\]
    \[\sqrt{a + b} \neq \sqrt{a} + \sqrt{b}\]
    \textbf{Pay attention to the last two. Don't make those mistakes!}

\subsection*{Simplifying Radicals}
    Let's say we wanted to simplify $\sqrt{250} + \sqrt{360}$.
    There's no obvious way to do this, but we can first \textbf{simplify} the two square roots.
    To simplify a square root, we first factor out the largest perfect square from the number inside it.
    We can factor $250$ into $25 \cdot 10$.
    Then, we use the rule of distributing exponents to get $\sqrt{250} = \sqrt{25 \cdot 10} = \sqrt{25}\sqrt{10} = 5\sqrt{10}$.
    Similarly, $\sqrt{360} = \sqrt{36 \cdot 10} = \sqrt{36}\sqrt{10} = 6\sqrt{10}$.
    Now we can add $5\sqrt{10}$ and $6\sqrt{10}$ to get $11\sqrt{10}$.
    When your answers contain square roots, you will generally be required to simplify them.

\end{document}
