\documentclass{article}

\usepackage{enumitem}

\title{1.4 Proportions and Fractional Reasoning}
\author{}
\date{}

\begin{document}
\maketitle
\section*{Key Concepts}
\subsection*{Percent Change}
Differences are sometimes expressed as percentages.
To do this, we take the ratio between the absolute difference and the \textbf{old} value and write it as a percentage.
For example, to express an increase from $15$ to $17$, we find the absolute difference, which is $17 - 15 = 2$, and we divide it by the old value to get $\frac{2}{15} \approx 0.133$.
Therefore, the increase is approximately $13.3\%$.
We can also subtract $1$ from the ratio between the new value and the old value, which again results in $\frac{17}{15} - 1 \approx 0.133 = 13.3\%$.
To apply a percent change to a value, we convert the percent into a ratio, add $1$, and then multiply it to the value.
For example, $a$ increased by $15\%$ would be $1.15a$, and $b$ decreased by $7\%$ would be $0.93b$.

A slightly counter-intuitive fact is that the inverse of a $x\%$ increase is NOT a $x\%$ decrease.
If I increase $4$ by $50\%$, I will get $6$.
But if I decrease $6$ by $50\%$, I will get $3$.
This is because the $50\%$ increase is calculated relative to $4$ while the $50\%$ decrease is relative to $6$.
For the same reason, percent changes don't simply add up.
If I increase $4$ by $50\%$, I will get $6$, and if I increase $6$ by $50\%$, I will get $9$.
This is not the same as increasing $4$ by $50\% + 50\% = 100\%$, which will result in $8$.
If you need to combine percent changes, you can first convert them to ratios.
A $50\%$ increase is the same as multiplying by $1 + 0.5 = 1.5$.
If I increased something by $50\%$ twice, it would be the same as multiplying by $1.5$ twice, which is equivalent to multiplying by $1.5^2 = 2.25$.
We can convert this back into a percent change to see that two $50\%$ increases is the same as a $125\%$ increase.

\subsection*{Proportionality}
\subsubsection*{Directly Proportional Relationships}
Two variables are said to be \textbf{directly proportional} if their ratio is constant.
In other words, $x$ and $y$ are directly proportional if $\frac{y}{x} = c$, where $c$ is some constant.
We can also write it as $y = cx$.
When two variables are directly proportional, changing one variable by some ratio changes the other variable by the same ratio.

\subsubsection*{Inversely Proportional Relationships}
Two variables are said to be \textbf{inversely proportional} if their product is constant.
In other words, $xy = c$ where $c$ is a constant, which can also be written as $y = \frac{c}{x}$.
Changing one variable by some ratio changes the other variable by the reciprocal of that ratio.

\subsection*{Proportions}
A \textbf{proportion} is an equation stating that two ratios are equal.
It looks like $\frac{a}{b} = \frac{c}{d}$.
When you know the values of three of the variables, you can solve for the fourth one using the usual methods.
When dealing with equations which have variables in the denominator, it's often useful to multiply both sides by the denominators containing the variables to eliminate the fractions.
\end{document}
