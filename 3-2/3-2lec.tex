\documentclass{article}

\usepackage[margin=0.5in]{geometry}
\usepackage{multicol}

\title{General Number Theory Lecture Problems}
\date{}
\author{}

\begin{document}
\maketitle
\begin{enumerate}
    \item Let $Z$ be a 6-digit positive integer, such as 247247, whose first three digits are the same as its last three digits taken in the same order. 
        What number must be a factor of $Z$?
        \vspace{3cm}
    \item What is the least positive two-digit integer that leaves a remainder of 3 when divided by
        each of the numbers 4, 5 and 6?
        \vspace{3cm}
    \item A number is called flippy if its digits alternate between two distinct digits. 
        For example, $2020$ and $37373$ are flippy, but $3883$ and $123123$ are not. 
        How many five-digit flippy numbers are divisible by $15?$
        \vspace{3cm}
    \item How many positive three-digit integers have a remainder of $2$ when divided by $6$, 
        a remainder of $5$ when divided by $9$, and a remainder of $7$ when divided by $11$?
        \vspace{3cm}
    \item Alex Zhang has $p$ pennies, $n$ nickels, $d$ dimes and $q$ quarters with a total value of $\$1.08$.
        If the numbers $p$, $n$, $d$ and $q$ are distinct and positive, and the greatest common divisor of each 
        pair of these numbers is $1$, what is the least possible value of $p + n + d + q$?
        \vspace{3cm}
\end{enumerate}
\end{document}
